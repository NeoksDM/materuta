\documentclass{article}
\usepackage[utf8]{inputenc}
\usepackage{amsmath}   
\usepackage{amssymb}  
\usepackage{amsthm}

\begin{document}

\section*{Recomendaciones de ChatGPT}
\subsection*{An\'alisis complejo}
\begin{enumerate}
	\item Complex Analysis - Lars Ahlfors
	\item Functions of One Complex Variable - John B. Conway
	\item Theory of Functions of a Complex Variable - Constantin Carath\'eodory
	\item Analytic Function Theory - Steven G. Krantz
	\item Complex Variables and Applications - James Ward Brown y Ruel V. Churchill
\end{enumerate}
\subsection*{Geometr\'ia Diferencial (sin variedades)}
\begin{enumerate}
	\item Differential Geometry of Curves and Surfaces - Manfredo P. do Carmo
	\item Elementary Differential Geometry - Barret O' Neill
	\item Lectures on Classical Differential Geometry - Dirk J. Struil
	\item Elementary Differential Geometry - Andrew Pressley
	\item Differential Geometry: A Frist Course - D. S. Struik
\end{enumerate}
\subsection*{Geometr\'ia Diferencial (con variedades)}
\begin{enumerate}
	\item Riemannian Geometry = Manfredo P. do Carmo
	\item Differential Gometry of Manifolds - Stephen T. Lovett
	\item Introduction to Smooth Manifolds - John M. Lee
	\item Foundations of Differential Geometry (Vol. 1 y 2) - Shoshichi Kobayashi y Katsumi Nomizu
	\item Riemannian Manifolds: An Introduction to Curvature - John M. Lee
\end{enumerate}
\subsection{\'Algebra Multilineal y Tensores}
\begin{enumerate}
	\item Multilinear Algebra - Werner Greub
	\item An Introduction to Tensors and Group Theory for Physicist - Nadir Jeevanjee
	\item Tensor Analysis on Manifolds - Richard L. Bishop y Samuel l. Goldberg
	\item Introduction to Tensor Analysis and the Calculus of Moving Surfaces - Pavel Grinfeld
	\item The Geometry of Physics: An introduction - Theodore Frankel
	\item Linear Algebra and Geomtry - P.K. Sutein, A. I. Kostrikin y Yu I. Manin
	\item Algebra - Serge Lang
	\item Abstract Algebra - David S. Dummit y Richard M. Foote
\end{enumerate}
\subsection{Geomter\'ia Diferencial y Tensores}
\begin{enumerate}
	\item Calculus on Manifolds - Michael Spivak
	\item Foundations of Differentiable Manifolds and Lie Groups - Frank W. Warner
	\item Riemannian Geometry - Manfredo P. do Carmo
	\item Tensor Calculus - Synge y A. Schild
\end{enumerate}
\subsection{L\'ogica Matem\'atica}
\begin{enumerate}
	\item Mathematical Logic - H.-D. Ebbinghaus, J. Flum, W. Thomas
	\item Mathematical Logic - Cory y Lascar
	\item Mathematical Logic - Stephen Cole Kleene
\end{enumerate}
\subsection{\'Algebra Abstracta}
\begin{enumerate}
	\item Abstract Algebra - Thomas W. Hungerford
	\item Abstract Algebra - David S. Dummit y Richard M. Foote
	\item Algebra - Serge Lang
	\item Topics in Algebra - I. N. Herstein
\end{enumerate}
\subsection{Topolog\'ia}
\begin{enumerate}
	\item Topology - James Munkres
	\item Algebraic Topology - Allen Hatcher
	\item Elements of Algebraic Topology - James R. Munkres
\end{enumerate}
\subsection{Teor\'ia de Conjuntos}
\begin{enumerate}
	\item Set Theory - Thomas Jech
\end{enumerate}
\subsection{An\'alisis Num\'erico}
\begin{enumerate}
	\item Numerical Analysis - Richard L. Burden y J. Douglas Faires
\end{enumerate}
\subsection{Estad\'istica}
\begin{enumerate}
	\item Mathematical Statistics and Data Analysis - John A. Rice
\end{enumerate}
\subsection{Probabilidad}
\begin{enumerate}
	\item Probability: Theory and Examples - Rick Durrett
	\item A First Course in Probability - Sheldon Ross
\end{enumerate}
\subsection{Teor\'ia de N\'umeros}
\begin{enumerate}
	\item An Introduction to the Theory of Numbers - G. H. Hardy y E. M. Wright
	\item Algebraic Number Theory - J\:urgen Neukirch
\end{enumerate}
\subsection{\'Algebra Lineal}
\begin{enumerate}
	\item Linear Algebra Done right = Sheldon Axler
	\item Matrix Analysis - Roger A. Horn y Charles R. Johnson
	\item Advanced Linear Algebra - Steven Roman
\end{enumerate}
\subsection{Geometr\'ia Euclidiana}
\begin{enumerate}
	\item Foundations of Geometry - David Hilbert
	\item Elements of Geometry - E.J Dijksterhuis
\end{enumerate}
\subsection{Geometr\'ia Anal\'itica}
\begin{enumerate}
	\item Analytic Geometry - Gordon Fuller y Dalton Tarwater
\end{enumerate}
\subsection{Trigonometr\'ia}
\begin{enumerate}
	\item Plane Trigonometry - S. L. Loney
\end{enumerate}
\subsection{Anal\'isis Matem\'atico}
\begin{enumerate}
	\item Calculus - Tom Apostol (Vol 1 y 2)
	\item Calculus - Michael Spivak
	\item Principles of Mathematical Analysis - Walter Rudin
\end{enumerate}
\subsection{Recursos Adicionales}
\begin{enumerate}
	\item An\'alisis Real y Complejo - Walter Rudin
	\item introduction to Smooth Manifolds - John M. Lee
\end{enumerate}

\section*{Ruta de Lectura Integrada}
\subsection*{Terna 1: Fundamentos}
\paragraph*{L\'ogica Matem\'atica}.
\begin{itemize}
	\item Mathematical Logic - H.-D. Ebbinghaus, J. Flum, W. Thomas
	\item Mathematical Logic - Cori y Lascar
	\item Mathematical Logic - Stephen Cole Kleene
\end{itemize}
\paragraph*{\'Algebra Abstracta}.
\begin{itemize}
	\item Abstract Algebra - Thomas W. Hungerford
	\item Abstract Algebra - David S. Dummit y Richard M. Foote
\end{itemize}
\paragraph*{\'Algebra Lineal}.
\begin{itemize}
	\item Linear Algebra Done Right - Sheldon Axler
	\item Linear Algebra and Geomtetry - P.K. Sutein, A. I. Kostrikin y Yu I. Manin
\end{itemize}
\subsection*{Terna 2: Geometr\'ia y Trigonometr\'ia}.
\paragraph*{Geometr\'ia Euclidiana}
\begin{itemize}
	\item Foundations of Geometry - David Hilbert
	\item Elements of Geometry - E. J, Dijksterhuis
\end{itemize}
\paragraph*{Geometr\'ia Anal\'itica}.
\begin{itemize}
	\item Analytic Geometry - Gordon Fuller y Dalton Tarwater
\end{itemize}
\paragraph*{Trigonometr\'ia}.
\begin{itemize}
	\item Plane Trigonometry - S. L. Loney
\end{itemize}
\subsection*{Terna 3: An\'alisis y Topolog\'ia}
\paragraph*{An\'alisis Matem\'atico (Introductorio)}.
\begin{itemize}
	\item Calculus - Michael Spivak
	\item Calculus - Tom Apostol (Vol. 1 y 2)
\end{itemize}
\paragraph*{Topolog\'ia}
\begin{itemize}
	\item Topology - James Munkres
\end{itemize}
\paragraph*{Teor\'ia de Conjuntos}
\begin{itemize}
	\item Set Theory - Thomas Jech
\end{itemize}
\subsection*{Terna 4: \'Algebra y Topolog\'ia Avanzada}
\paragraph*{\'Algebra Abstracta Avanzada}
\begin{itemize}
	\item Algebra - serge Lang
	\item Topics in Algebra - I. N. Hestein
\end{itemize}
\paragraph{Topolog\'ia Algebraica}
\begin{itemize}
	\item Algebraic Topology - Allen Hatcher
	\item Elements of Algebraic Topology - James R. Munkres
\end{itemize}
\paragraph{\'Algebra Lineal Avanzada}
\begin{itemize}
	\item Matrix Analysis - Roger A. Horn y Charles R. Johnson
	\item Advanced Linear Algebra - Steven Roman
\end{itemize}
\subsection*{Terna 5: An\'alisis Avanzado y Geometr\'ia Diferencial}
\paragraph*{An\'alisis Matem\'atico (Avanzado)}.
\begin{itemize}
	\item Mathematical Analysis - Tom Apostol
	\item Principles of Mathematical Analysis - Walter Rudin
\end{itemize}
\paragraph*{Geometr\'ia Diferencial (sin variedades)}.
\begin{itemize}
	\item Elementary Differential Geometr y - Barret O' Neill
	\item Differential Geometry: A First Course - D. S. Struik
	\item Lectures on Classical Differential Geometry - Dirk J. Struik
	\item Elementary Differential Geometry - Andrew Pressley
	\item Differential Geometry - Erwin Kreyszig
	\item Differential Geometry and Its Applications - John Oprea
	\item Differential Geometry - Kristopher Tapp
	\item Differential Geometry of Curvos ans Surfaces - Manfredo P. do Carmo
	\item Differential Geometry - William C. Graustein
\end{itemize}
\paragraph*{An\'alisis Complejo}
\begin{itemize}
	\item Complex Variables And Applications - James Ward Brown y Ruel V. Churchill
	\item Complex Analysis - Lars Ahlfors
	\item Theory of Functions of a Complex Variable - Constantin Caratheodory
	\item Analytic Function theory - Steven G. Krantz
	\item Functions of One Complex Variable I - John B. Conway
	\item Complex Analysis - Elias M. Stein y RAmiShakarchi
	\item Real and Complex Analysis - Walter Rudin
	\item Function Theory of One Complex Variable - Robert E. Greene & Steven G. Krantz
\end{itemize}
\paragraph*{\'Algebra Multilineal y Tensores}.
\begin{itemize}
	\item Linear Algebra and Geometry - P.K. Suetin, A. I. Kostrikin y Yu I. Manin
	\item Multilinear Algebra - Werner Greub
	\item Tensor Calculus - John Lighton Synge & Alfred Schild
	\item Introduction to Tensor Analysis and the Calculus of Moving Surfaces - Pavel Grinfeld
	\item An Introduction to Tensors and Group Theory for Physicists - Nadir Jeevanjee
	\item Algebra - Serge Lang
\end{itemize}
\subsection*{Terna 6: Geometr\'ia Diferencial con Variedades y An\'alisis Complejo}
\paragraph*{Geometr\'ia Diferencial y Tensores}
\begin{itemize}
	\item Calculus on Manifolds - Michael Spivak
	\item Tensor Analysis on Manifolds - Richard L. Bishop y Samuel I. Goldberg
	\item Foundations of Differentiable Manifolds and Lie Groups - Frank W. Warner
	\item Tensor Calculus - Synge A. Schild
	\item The Geometry of Physcics: An Introduction - Thedore Frankel
\end{itemize}
\paragraph*{Geometr\'ia Diferencial con Variedades}
\begin{itemize}
	\item Introduction to Smooth Manifolds - John M. Lee
	\item Differential Geometry of Manifolds - Stephen T. Lovett
	\item Differentail Topology - Victor Guillemin & Alan Pollack
	\item Riemannian Geometry - Manfredo P. do Carmo
	\item Riemannian Manifolds: An Introduction to Curvature - John M. Lee
	\item Foundations of Differential Geometry (Vol. 1 y 2) - Shoshichi Kobayashi y Katsumi Nomizu
\end{itemize}
\subsection*{Estad\'istica, Probabilidad y An\'alisis Num\'erico}
\paragraph*{Estad\'istica}.
\begin{itemize}
	\item Mathematical Statistics and Data Analysis - John A. Rice
\end{itemize}
\paragraph*{Probabilidad}.
\begin{itemize}
	\item Probability: Theory and Examples - Rick Durrett
	\item A First Course in Probability - Sheldon Ross
\end{itemize}
\paragraph*{An\'alisis Num\'erico}
\begin{itemize}
	\item Numerical Analysis - Richard L. Burden y J. Douglas Faires
\end{itemize}
\subsection*{Teor\'ia de N\'umeros y Adicionales}
\paragraph*{Teor\'ia de N\'umeros}
\begin{itemize}
	\item An Introduction to the Theory of Numbers - G. H. Hardy y E. M. Wright
	\item Algebraic Number Theory - J\"urgen Neukirch
\end{itemize}












\section*{Provisional}
\begin{enumerate}
\item Sistemas expertos.
Los sistemas expertos son un subcampo de la inteigencia artificial (IA) que se centra en emular el razonamiento humano experto para resolver problemas espec[ificos en dominios concretos. Para entenderlos bien es necesario tener una base s\'olida en matem\'aticas y ciencias computacionales.
\begin{center}
Fundamentos de los sistemas expertos
\end{center}
\begin{itemize}
\item Los sistemas expertos se basan en la l\'ogica y el razonamiento deductivo. Comprender la l\'ogica proposicional y la l\'ogica de predicados es esencial. Libros recomendados 
\begin{itemize}
\item Logic A Very Short Introduction - Graham Priest
\item Mathematical logic - Cori Y Lascar Volumen 1
\item Mathematical Logic - Kleene Vol\'umenes 1 y 2
\item Mathematical Logic - Cori y Lascar
\end{itemize}
\end{itemize}

\item L\'ogica difusa
\item Algoritmos gen\'eticos
\item Programaci\'on evolutiva
\item M\'aquinas de vectores de soporte (SVM)
\item Redes bayesianas
\item Modelos markovianos
\item \'Arboles de decisi\'on
\item Regresi\'on lineal
\item Algoritmo de agrupamiento (clusterin)
	\item Sistemas de inferencia borrosa
	\item Redese neuornales bayesianas
	\item Algoritmos de b\'usqueda heur\'istica
	\item Sistema multiagentes
	\item Algoritmos de optimizaci\'on
	\item Algoritmos de aprendizaje autom\'atico no supervisado
	\item Algoritmos de aprendizaje autom\'atico superivaso sin usar redes neuronales
	\item Programaci\'on simb\'olica
	\item Redes de Petri
	\item Sistemas de producci\'on 
	\item Aprendizaje basado en instancias
	\item Sistemas de recomendaci\'on basados en filtrado colaborativo
	\item Sistemas basados en conocimiento
	\item Aprendizaje autom\'atico basado en reglas
	\item Algoritmos de clasificaci\'on no param\'etricos
\end{enumerate}
Los sistemas expertos son un tipo de sistema de inteligencia artificial que imita el proceso de toma de decisiones de un experto humano en un dominio específico. Estos sistemas se utilizan para resolver problemas complejos mediante el razonamiento basado en reglas y el conocimiento almacenado. Aquí te proporcionaré una explicación exhaustiva sobre los sistemas expertos, así como algunas recomendaciones de libros matemáticamente rigurosos para matemáticos y científicos computacionales.

 Definición y Funcionamiento de los Sistemas Expertos

Los sistemas expertos son programas de computadora diseñados para imitar la capacidad de un experto humano en un campo particular. Estos sistemas utilizan conocimiento experto, reglas de inferencia y técnicas de razonamiento para resolver problemas complejos en áreas como la medicina, la ingeniería, las finanzas, entre otras.

 Componentes de un Sistema Experto:

1. **Base de Conocimiento**: Contiene el conocimiento experto en forma de reglas, hechos y relaciones.
   
2. **Motor de Inferencia**: Es la parte del sistema que aplica las reglas de inferencia para llegar a conclusiones lógicas a partir de los datos de entrada.

3. **Interfaz de Usuario**: Permite la comunicación entre el usuario y el sistema experto, facilitando la entrada de datos y la presentación de resultados.

 Libros Recomendados para Matemáticos:

1. **"Introducción a la Lógica Matemática"** de Elliott Mendelson: Este libro proporciona una introducción sólida a la lógica matemática, que es fundamental para comprender el razonamiento detrás de los sistemas expertos.

2. **"Teoría de Conjuntos"** de Kenneth Kunen: La teoría de conjuntos es esencial en la construcción de sistemas formales. Este libro es una referencia estándar en el campo y proporciona una visión profunda de los fundamentos de la teoría de conjuntos.

3. **"Lógica Inductiva y Probabilidades"** de Richard T. Cox: Este libro aborda la lógica inductiva y las probabilidades, que son importantes en la toma de decisiones dentro de los sistemas expertos.

 Libros Recomendados para Científicos Computacionales:

1. **"Introducción a los Algoritmos"** de Thomas H. Cormen, Charles E. Leiserson, Ronald L. Rivest, y Clifford Stein: Este libro es fundamental para comprender los algoritmos y estructuras de datos utilizados en la implementación de sistemas expertos.

2. **"Introducción a la Inteligencia Artificial"** de Stuart Russell y Peter Norvig: Proporciona una visión general de los fundamentos de la inteligencia artificial, incluidos los sistemas expertos, con un enfoque en los aspectos algorítmicos y computacionales.

3. **"Procesamiento del Lenguaje Natural con Python"** de Steven Bird, Ewan Klein, y Edward Loper: Dado que muchos sistemas expertos requieren comprender y generar lenguaje natural, este libro es una excelente introducción práctica al procesamiento del lenguaje natural, utilizando el lenguaje de programación Python.

 Recomendaciones Adicionales:

- **"Sistemas Expertos: Principios y Programación"** de Giarratano y Riley: Este libro ofrece una introducción práctica a la construcción de sistemas expertos, cubriendo desde los fundamentos hasta la implementación práctica.

- **"Principios de Diseño de Sistemas Expertos"** de Peter Lucas, Luis Guerrero, y Antonio Bahamonde: Explora los principios fundamentales del diseño de sistemas expertos, centrándose en aspectos como la representación del conocimiento y el razonamiento.

- **"Inteligencia Artificial: Una Perspectiva Moderna"** de Stuart Russell y Peter Norvig: Aunque no se centra exclusivamente en sistemas expertos, este libro proporciona una amplia cobertura de la inteligencia artificial, incluidos los enfoques basados en el conocimiento.

Estos libros proporcionan una sólida base matemática y computacional para comprender y diseñar sistemas expertos, desde los principios teóricos hasta la implementación práctica. Al estudiar estos recursos, los matemáticos y científicos computacionales pueden adquirir las habilidades necesarias para trabajar en el campo de la inteligencia artificial y los sistemas expertos con rigor y profundidad.
\end{document}

